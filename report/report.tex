\documentclass[a4paper,12pt]{article}

\usepackage[utf8]{inputenc}
\usepackage{amsmath}
\usepackage{graphicx}
\usepackage{hyperref}
\usepackage{geometry}
\geometry{a4paper, margin=1in}

\title{Artificial Intelligence Project Report}
\author{Your Name}
\date{\today}

\begin{document}

\maketitle

\begin{abstract}
This report provides an overview of the project conducted in the Artificial Intelligence course. The project focuses on [brief description of the project].
\end{abstract}

\tableofcontents

\section{Introduction}
Provide an introduction to the project, including the objectives and the significance of the project.

\section{Background}
Discuss the background information and related work that is relevant to the project.

\section{Methodology}
Describe the methods and techniques used in the project. Include any algorithms, models, or tools that were utilized.

\subsection{Minimax Algorithm}
The Minimax algorithm is a decision-making algorithm used in artificial intelligence, particularly in game theory and two-player games such as chess, tic-tac-toe, and checkers. The algorithm aims to minimize the possible loss for a worst-case scenario. When dealing with gains, it is referred to as "maximin"—to maximize the minimum gain.

The Minimax algorithm operates by simulating all possible moves in a game and their subsequent outcomes, assuming that both players play optimally. The algorithm consists of the following steps:

1. **Generate the Game Tree**: Create a tree structure where each node represents a game state, and each edge represents a possible move by a player.

2. **Evaluate Terminal States**: Assign a value to each terminal state (leaf node) based on the outcome of the game (win, lose, draw).

3. **Backpropagate Values**: Starting from the terminal states, propagate the values back up the tree. For each non-terminal node:
   - If it is the maximizing player's turn, assign the maximum value of its children.
   - If it is the minimizing player's turn, assign the minimum value of its children.

4. **Choose the Optimal Move**: At the root node, the maximizing player chooses the move that leads to the child node with the highest value.

The Minimax algorithm can be enhanced with alpha-beta pruning, which reduces the number of nodes evaluated in the game tree by eliminating branches that cannot influence the final decision.

Here is a pseudocode representation of the Minimax algorithm:

\begin{verbatim}
function minimax(node, depth, maximizingPlayer):
    if depth == 0 or node is a terminal node:
        return the heuristic value of node

    if maximizingPlayer:
        maxEval = -inf
        for each child of node:
            eval = minimax(child, depth - 1, false)
            maxEval = max(maxEval, eval)
        return maxEval
    else:
        minEval = +inf
        for each child of node:
            eval = minimax(child, depth - 1, true)
            minEval = min(minEval, eval)
        return minEval
\end{verbatim}

In this project, the Minimax algorithm was implemented to [describe the specific application of the algorithm in your project, e.g., solve a game, make decisions, etc.]. The algorithm was tested and evaluated based on [describe the evaluation criteria and results].

\section{Results}
Present the results of the project. Include any data, graphs, or figures that help to illustrate the findings.

\section{Discussion}
Interpret the results and discuss their implications. Mention any limitations of the study and potential areas for future work.

\section{Conclusion}
Summarize the key findings of the project and their significance.

\section{References}
\bibliographystyle{plain}
\bibliography{references}

\end{document}